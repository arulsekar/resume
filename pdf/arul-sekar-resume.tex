\documentclass[letter,10pt]{article}
%\documentclass[a4paper,10pt]{article}

%A Few Useful Packages
\usepackage{marvosym}
\usepackage{fontspec} 					%for loading fonts
\usepackage{xunicode,xltxtra,url,parskip} 	%other packages for formatting
\RequirePackage{color,graphicx}
\usepackage[usenames,dvipsnames]{xcolor}
%\usepackage[big]{layaureo} 				%better formatting of the A4 page
% an alternative to Layaureo can be ** \usepackage{fullpage} **
\usepackage[cm]{fullpage}
\usepackage{supertabular} 				%for Grades
\usepackage{titlesec}					%custom \section

%Setup hyperref package, and colours for links
\usepackage{hyperref}
\definecolor{linkcolour}{rgb}{0,0.2,0.6}
\hypersetup{colorlinks,breaklinks,urlcolor=linkcolour, linkcolor=linkcolour}

%FONTS
\defaultfontfeatures{Mapping=tex-text}
\setmainfont[SmallCapsFont = Fontin SmallCaps]{Fontin}

%CV Sections inspired by: 
%http://stefano.italians.nl/archives/26
\titleformat{\section}{\Large\scshape\raggedright}{}{0em}{}[\titlerule]
\titlespacing{\section}{0pt}{3pt}{3pt}
%Tweak a bit the top margin
%\addtolength{\voffset}{-1.3cm}

%Italian hyphenation for the word: ''corporations''
\hyphenation{im-pre-se}

%-------------WATERMARK TEST [**not part of a CV**]---------------
\usepackage[absolute]{textpos}
\usepackage{fancyhdr}

\pagestyle{fancy}
\fancyhf{}
 
%\cfoot{\footnotesize{Page \thepage \hspace{1pt} of \pageref{LastPage}}}
\cfoot{\footnotesize{Page \thepage \hspace{1pt} of 2}}
\rfoot{\footnotesize{\href{http://www.arulsekar.com}{www.arulsekar.com}}}
\lfoot{\textsc{\footnotesize{Arul Sekar}}}
\renewcommand{\headrulewidth}{0pt}

\setlength{\TPHorizModule}{30mm}
\setlength{\TPVertModule}{\TPHorizModule}
\textblockorigin{2mm}{0.65\paperheight}
\setlength{\parindent}{0pt}

%--------------------BEGIN DOCUMENT----------------------
\begin{document}

% \pagestyle{empty} % non-numbered pages

\font\fb=''[cmr10]'' %for use with \LaTeX command

%--------------------TITLE-------------
\par{\centering
		{\Huge \textsc{Arul Selvan Sekar}
	}\bigskip\par}

\begin{center}
\begin{tabular}{ rl | rl | rl }
	\textsc{       email:}       & \href{mailto:arul@arulsekar.com}{arul@arulsekar.com} &
	\textsc{       Phone:}     & 425-381-6841 &
	\textsc{       Website:}   & \href{http://www.arulsekar.com}{www.arulsekar.com}
\end{tabular}
\end{center}

%--------------------SECTIONS-----------------------------------

\section{Work Experience}
\begin{tabular}{r|p{16cm}}

\emph{Current} & \textbf{Senior Software Engineer, TEGRA}\\
\textsc{Jan 2013}&\emph{NVIDIA, Redmond WA}\\
&\footnotesize{\textbf{Core Program Architect, Automotive Foundation (Current)
}}\\
&\footnotesize{
-- Responsible for architecture of SW Core program for Automotive Foundation (Flashing, Bootloader, TEE, Comms, Platform, etc.)
}\\
&\footnotesize{
-- Architect system level requirements and feature interdependencies, and decompose them to various components
}\\
&\footnotesize{
-- Coordinate Feature Architects to complete architecture of component level requirements
}\\
&\footnotesize{\textbf{Software Architecture and Development, Automotive
}}\\
&\footnotesize{
-- Design and implementation of system image generation and flashing components
}\\
&\footnotesize{
-- PSIRT PIC and Safety PIC for above components
}\\
&\footnotesize{
-- Architectural lead for next iteration of modular and scalable flashing framework
}\\
&\footnotesize{
-- Implement suspend framework for QNX OS running on hypervisor environment
}\\
&\footnotesize{
-- Requirement and design specifications for HDCP repeater use-cases
}\\
&\footnotesize{\textbf{Tablet SW Architecture, Android Power and Perf
}}\\
&\footnotesize{
 -- Improve software architecture for memory management, performance and power governance
}\\
&\footnotesize{
 -- Modifications to App Framework (Java), Native Libraries (C++), Kernel (C) and HAL 
}\\
&\footnotesize{
 -- Create metrics and viable solutions to quantify and track these improvements
}\\
&\footnotesize{
-- Define architectural changes, improvements and metrics for upcoming products
}\\
&\footnotesize{\textbf{System Software Development, Android
}}\\
&\footnotesize{
 -- Kernel and user-space software stack that balance power and performance of SoC
}\\
&\footnotesize{
 -- TEE, Secure OS on SoC for Security operations (such as RSA, AES, HMAC, CMAC) 
}\\
&\footnotesize{\textbf{System Software Development, Windows on ARM
}}\\
&\footnotesize{
 -- Power management for USB, XUSB software stack
}\\
&\footnotesize{
 -- POR for XUSB changes on upcoming SoC 
}\\
&\footnotesize{
 -- Customer interaction for debugging on upcoming products 
}\\

\multicolumn{2}{c}{} \\

\emph{Dec 2012} & \textbf{Embedded Software Engineer, OMAP Platform Business Unit}\\
\textsc{Jun 2012}&\emph{Texas Instruments, Redmond WA}\\
&\footnotesize{
 - Responsible for developing OS and firmware drivers on Windows RT (WoA) tablets based on the OMAP processor
}\\
&\footnotesize{
 - Owner of Variable Services component of Security; point of contact in TI Redmond for security-related issues
}\\
&\footnotesize{
 - Ownership of functionality, robustness, and performance of UEFI and OS driver, and Trusted Application in ARM TrustZone 
}\\
&\footnotesize{
 - Development of drivers from mid-development level to production level, including passing the Windows certification
}\\
&\footnotesize{
 - Active discussions and communications with numerous TI partners that provide implementation and specifications
}\\
&\footnotesize{
 - Working knowledge of security including ARM TrustZone, secure boot, measure boot, and trusted applications
}\\
&\footnotesize{
 - Fluent in UEFI DXI drivers and TianoCore EDKII; Familiarity of WHCK Security tests
}\\
&\footnotesize{
 - Participate in design and architecture reviews for Security
}\\
&\footnotesize{
 - Position requires usage of Lauterbach JTAG for HW debugging, WinDbg for debugging Windows device drivers
}\\
&\footnotesize{
 - Recent part of Power Management team, resolving related issues and bugs related to PMIC and PRCM
}\\

\multicolumn{2}{c}{} \\

\textsc{Jun 2012} & \textbf{Teaching Assistant, Electrical Engineering Dept.}\\
\textsc{Sep 2010}&\emph{University of Washington, Seattle}\\
&\footnotesize{
 - EE 478 (Embedded Capstone), EE 472 (Microcomputer Systems), and EE 271 (Digital Circuits and Systems)
}\\
&\footnotesize{
 - Supervise, guide, and evaluate students with labs and final projects; review sessions for additional materials
}\\
&\footnotesize{
 - Critique and award submitted functional and design specifications, proposals, and system designs
}\\

\multicolumn{2}{c}{} \\


\textsc{Sep 2008} & \textbf{Engineering Intern, Premium Applications Engineering}\\
\textsc{Sep 2010}&\emph{ARRIS Group Inc./Digeo Inc., Kirkland WA}\\
&\footnotesize{
 - Work as developer in Software Development Team for the Moxi HD DVR and Moxi Mate devices
}\\
&\footnotesize{
 - Fixed bugs related to the Moxi C++ Applications and XML/C++ Framework for UI and data layers
}\\
&\footnotesize{
 - Filter performance and warning bugs generated by Coverity and filed the appropriate bugs on Bugzilla
}\\

\end{tabular}

\section{Education}
\begin{tabular}{rl}	
 \textsc{June} 2012 & Master of Science, \textsc{Electrical Engineering}, \textbf{University of Washington}, Seattle\\
&\normalsize {Concentration}: Robotics \& Controls and Embedded Systems \\
& \small Overall GPA: \normalsize\textbf{3.75} \\
&\\

\textsc{August} 2009 & Bachelor of Science, \textsc{Electrical Engineering}, \textbf{University of Washington}, Seattle \\
&\normalsize {Concentration}: Embedded Systems \\
& \small\emph{Graduated Cum Laude} | Overall GPA: \normalsize\textbf{3.76} \small | Major GPA: \normalsize\textbf{3.94} \\

\end{tabular}


\section{Academic Achievements}
\begin{tabular}{r|p{16cm}}

\textsc{Jun 2012} & \textbf{Dynamic Gravity Compensation for Raven II Surgical Robot}\\
\textsc{Jan 2011}&\emph{BioRobotics Laboratory, University of Washington, Seattle}\\
&\footnotesize{
 - Analyze the effects on dynamics due to change in orientation for Raven Surgical Robot
}\\
&\footnotesize{
 - Develop add-on hardware to calculate orientation of robot using sensor measurements
}\\
&\footnotesize{
 - Communicate orientation data to control system and modify the algorithm to compensate
}\\

\multicolumn{2}{c}{} \\

\textsc{Spring} & \textbf{RSK Robotic Arm}\\
\textsc{2011}&\emph{Capstone Project for EE 449 (Control System Design)}\\
&\footnotesize{
 - 3-DOF arm that aids powered wheelchair users to automatically activate handicap door buttons upon request
}\\
&\footnotesize{
 - Implemented with cost-effective hardware, GUI for end-user and designer; using threads and wxWidgets on Linux
}\\
&\footnotesize{
 - Computer vision and Inverse Kinematics calculations for positional control; Safety using dynamic velocity control
}\\

\multicolumn{2}{c}{} \\

\textsc{Winter} & \textbf{Lunar Rover Prototype}\\
\textsc{2011}&\emph{Project Manager for EE 542, Rocket City Space Pioneers' Google Lunar X Prize}\\
&\footnotesize{
 - Developed Functional, Requirement, Architecture, and Test Specifications, HW/SW Implementation documents
}\\
&\footnotesize{
 - For design of control system on lunar rover, covering locomotion, camera, communication with satellite, etc.
}\\
&\footnotesize{
 - Involved in project planning and timelines, and design discussion with team and customers
}\\
\multicolumn{2}{c}{} \\


\textsc{Fall} & \textbf{Small Scale Positioning System}\\
\textsc{2008} &\emph{Capstone Project for EE 478 (Design of Computer Subsystems)}\\
&\footnotesize{
 - Portable device that enables the user to track the 3D location of any targeted object indoors with high accuracy
}\\
&\footnotesize{
 - Consists of six independent subsystems that coordinate, and use concepts of trilateration, to calculate location
}\\
&\footnotesize{
 - Wireless communication and high-level power management to preserve battery life on the subsystems
}\\

\end{tabular}

\section{Highlighted Skills}

\begin{tabular}{r|p{16cm}}

\textsc{Hardware} & \emph{Processors: } Tegra, OMAP, PIC, MSP | \emph{Devices: } Android Tablet, Win RT Tablet, Automotive AI Platform \\%\multicolumn{2}{c}{}\\
\textsc{Languages} & C, C++, Python, Verilog, MATLAB | \emph{Dev Stack: } Android, Windows 8 RT, Linux, QNX, UEFI, ROS, Bare-metal\\%\multicolumn{2}{c}{}

\end{tabular}

\end{document}
